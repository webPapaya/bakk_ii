\section*{Kurzfassung}
\vspace{0.5cm}

Die vorliegende Arbeit beschäftigt sich mit dem Thema \emph{Concurrency in Ruby}. Dabei werden Grundlegende Konzepte für gleichzeitige und parallele Programmierung vorgestellt. Im weiteren Verlauf dieser Arbeit werden diese Konzepte in der Programmiersprache Ruby umgesetzt. 

Die Programmiersprache Ruby ist nicht für ihre Concurrency Features bekannt. Der Mythos das parallele Programmierung in Ruby unmöglich ist hält sich bis heute. Lediglich eine Implementierung, Ruby MRI, besitzt einen GIL (Global Interpreter Lock) der tatsächlich das parallele Ausführen von Ruby unterbindet. Dennoch kann auch Ruby MRI mehrere Operationen gleichzeitig behandeln.  Andere Interpreter wie Rubinius oder JRuby besitzten aufgrund deren Architektur keinen GIL und können daher Code problemlos parallel ausführen. Anhand von Beispielen werden Probleme erklärt welche im Zusammenhang mit Concurrency und paralleler Programmierung entstehen. 

Zu Beginn dieser Arbeit wird ein genereller Überblick über das Thema Concurrency und parallele Programmierung  gegeben. Im zweiten Teil werden Design Pattern vorgestellt welche mit dem Thema der Arbeit in Zusammenhang stehen. 

Im zweiten Teil dieser Arbeit wird speziell auf die Programmiersprache Ruby und deren Concurrency Model eingegangen. Im letzten Teil werden Referenzimplementierungen der Design Patterns in Ruby vorgestellt. Zum Abschluss wird ein persönliches Fazit gezogen.


\paragraph{Schlagwörter: Ruby, Concurrency, Reactor Pattern, Futures, Actor Based Model, Rubinius, Ruby, MRI}