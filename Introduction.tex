\section{Einleitung}
\label{section:Einleitung}

\subsection{Concurrency}
\label{section:concurrency}


Concurrency sind Activitäten die zur selben Zeit auf einem Computer ausgeführt werden. Diese Aktivitäten können in unterschiedliechen Umgebungen ausgeführt werden \cite[p. 14]{Erb2012}:

\begin{itemize}
  \item Single Core Prozessoren
  \item Multi Core Prozessoren
  \item Multi Prozessoren
  \item Unterschiedliche Maschinen in einem Distributed System
\end{itemize}

Die Motivation hinter Concurrency dienen Hauptsächlich einer Steigerung der Performance einer Application. Diese Steigerungen können in drei Kategorien unterteilt werden \cite[p. 18]{Can08}: 

\emph{Reduzierte Latenzen}
	Eine Aufgabe wird in schneller ausgeführt indem sie in kleinere Aufgaben unterteilt wird, welche gleichzeitig abgearbeitet werden können. \cite[p. 18]{Can08}
    
\emph{Latenzen verstecken}
	Lang andauernde Aufgaben werden gebündelt und abgearbeitet. Gleichzeitig können andere Aufgaben abgearbeitet werden bis die lang andauernden Aufgaben erledigt sind. Dies kann besonders bei langandauernden Netzwerk oder Festplatten Operationen sinnvoll sein da diese ansonsten den Programfluss blockeren.
\emph{Die Menge an verarbeiteten Daten pro Zeiteinheit zu veringern}
	Wenn mehrere Tasks gleichzeitig Ausgeführt werden kann die Menge an verarbeiten Daten erhöht werden.	

In verteilten Systemen wie Web Applikationen ist die parallele Ausführung ein essentieller Bestandteil \cite[p. 14]{Erb2012}. Dabei können jedoch jene Teile eines Systems sequentiell abgearbeitet werden, welche keinen gemeinsamen Status besitzen. Teile des Programms welche einen gemeinsamen Status besitzen müssen sich über Concurrency gedanken machen. \cite[p. 18]{Can08}

Ein konkretes Beispiel dazu bietet ein Webserver. Mehrere Anfragen werden von unterschiedlichen Clients an einen HTTP-Webserver gesendet, welcher die Anfragen gleichzeitig verarbeitet. Der Webserver leitet die Anfragen an ein das zugehörige Program weiter welches den jeweiligem HTTP-Response sequentiell generiert. \cite[p. 18]{Can08}


\subsection{Parallelismus}
\label{section:Parallelismus}

Parallelismus ist eine Eigenschaft von Software, bei der mindestens zwei Threads (TODO: Describe Threads) Code gleichzeitig ausführen. 


\subsection{Concurrency vs. Parallelismus}
\label{section:Parallelismus}

In der Literatur finden sich unterschiedliche Definitionen über Parallelismus und Concurrency. Oft wird zwischen diesen beiden Definitionen jedoch nicht unterschieden.

Wenn zwei Aufgaben gleichzeitig Ausgeführt werden gibt es mehrere Möglichkeiten in welcher diese ausgeführt werden \cite[p. 14]{Erb2012}:

\begin{itemize}
  \item Sie werden sequenziell abgearbeitet (die Reihenfolge spielt dabei keine Rolle)
  \item Sie werden abwechseln abgearbeitet
  \item Sie werden parallel abgearbeitet
\end{itemize}


Oracle beschreibt Parallelismus als eine Kondition die auftritt wenn mindestens zwei Threads gleichzeitig ausgeführt werden. Concurrency wird als eine generalisierte Form von Parallelismus verstanden bei der zwei Threads einen fortschritt machen. Dies beinhaltet auch virtuellen Parallelismus bei dem Threads durch Time-Slicing(TODO: Describe Time-Slicing) Code abwechselnd ausführen \cite[]{oracle:multithreading}. Durch Time-Slicing kann eine pseudo Parallelität auf einem einzelnen Prozessor erreicht werden. Durch diese Definition schließt sich dass eine richtige Parallelität nur dann existiert wenn mehrere Aufgaben auf der selben Anzahl an Prozessorkernen ausgeführt werden. (TODO: Hannes fragen ob dieser Schluss stimmt)

In dieser Thesis wird von Parallelität gesprochen wenn zwei Aufgaben auf der CPU parallel abgearbeitet werden. Von Concurrency wird in dieser Thesis gesprochen wenn mindestens zwei Tasks auf der CPU sequenziell oder abwechselnd abgearbeitet werden.


\subsection{Asynchrone Programmierung}
(TODO: write Section)

\subsection{Event getriebene Programmierung}
(TODO: write Section)
(TODO: write about dispatching and demultiplexing)
(TODO: write about callback functions)


\subsection{I/O Operations}
(TODO: write Section)
(TODO: add statistiks about Disk I/O, Network I/O, Memory I/O)