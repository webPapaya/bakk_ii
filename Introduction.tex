\section{Einleitung}
\label{section:Einleitung}


\subsection{Performance}
(TODO: was ist Performance)

\subsection{Concurrency}
\label{section:concurrency}

Concurrency sind Aktivitäten die zur selben Zeit auf einem Computer ausgeführt werden. Diese können in unterschiedliechen Umgebungen ausgeführt werden \cite[p. 14]{Erb2012}:

\begin{itemize}
  \item Single Core Prozessoren
  \item Multi Core Prozessoren
  \item Multi Prozessoren
  \item Unterschiedliche Maschinen in einem Distributed System
\end{itemize}

Das gleichzeitige Ausführen von Operationen kann zu einer Steigerung der Performance einer Applikation führen. Die Steigerung von Performance kann in drei Kategorien unterteilt werden \cite[p. 18]{Can08}:

\emph{Reduzierte Latenzen}
	Eine Aufgabe wird in schneller ausgeführt indem sie in kleinere Aufgaben unterteilt wird, welche gleichzeitig abgearbeitet werden können. \cite[p. 18]{Can08}

\emph{Latenzen verstecken}
	Lang andauernde Aufgaben werden gebündelt und abgearbeitet. Gleichzeitig können andere Aufgaben abgearbeitet werden bis die lang andauernden Aufgaben erledigt sind. Dies kann besonders bei Netzwerk oder Festplatten Operationen sinnvoll sein da diese ansonsten den Programfluss blockeren können.

\emph{Die Menge an verarbeiteten Daten pro Zeiteinheit zu veringern}
	Wenn mehrere Tasks gleichzeitig Ausgeführt werden kann die Menge an verarbeiten Daten pro Zeiteinheit erhöht werden.	

In verteilten Systemen wie Web Applikationen ist die parallele Ausführung ein essentieller Bestandteil \cite[p. 14]{Erb2012}. 

Teile einer Applikation welche keinen gemeinsamen Status besitzen und sequentiell abgearbeitet werden können mit wenig Aufwand parallel ausgeführt werden. \cite[p. 18]{Can08}

Ein konkretes Beispiel dazu bietet ein Webserver. Mehrere Anfragen werden von unterschiedlichen Clients an einen HTTP-Webserver gesendet, welcher die Anfragen gleichzeitig verarbeitet. Der Webserver leitet die Anfragen an ein das zugehörige Program weiter welches den jeweiligem HTTP-Response sequentiell generiert. \cite[p. 18]{Can08}

Für einen modernen Webserver müssen folgende Bedingungen erfüllt sein \cite[p. 2]{Sch97}: 

\emph{Concurrency}
	Ein Server muss mehrere Anfragen von Clients gleichzeitig verarbeiten können. 

\emph{Efficiency}
	Der Server muss Latenzen gering halten und die CPU nicht unnötig belasten. 

\emph{Programming simplicity}
	Der Server muss einfache Schnittstellen für Concurrent Programming bieten (TODO: concurrent programming?)

\emph{Adaptability}
	Neue Protokolle wie HTTP 2.0 sollten mit minimalen Implementierungsaufwand realisierbar sein.



\subsection{Parallelismus}
\label{section:Parallelismus}

Parallelismus ist eine Eigenschaft von Software, bei der mindestens zwei Threads (TODO: Describe Threads) Code gleichzeitig ausführen. 


\subsection{Concurrency vs. Parallelismus}
\label{section:Parallelismus}

Wenn zwei Aufgaben gleichzeitig Ausgeführt werden gibt es mehrere Möglichkeiten wie diese ausgeführt werden \cite[p. 14]{Erb2012}:

\begin{itemize}
  \item Sie werden sequenziell abgearbeitet (die Reihenfolge spielt dabei keine Rolle)
  \item Sie werden abwechseln abgearbeitet
  \item Sie werden parallel abgearbeitet
\end{itemize}


In der Literatur finden sich unterschiedliche Definitionen über Parallelismus und Concurrency. Oft wird zwischen diesen beiden Definitionen jedoch nicht unterschieden.

Oracle beschreibt Parallelismus als eine Kondition die auftritt wenn mindestens zwei Threads gleichzeitig ausgeführt werden. Concurrency wird als eine generalisierte Form von Parallelismus verstanden bei der zwei Threads einen fortschritt machen. Dies beinhaltet auch virtuellen Parallelismus bei dem Threads durch Time-Slicing(TODO: Describe Time-Slicing) Code abwechselnd ausführen \cite[]{oracle:multithreading}. Durch Time-Slicing kann eine pseudo Parallelität auf einem einzelnen Prozessor erreicht werden. Durch diese Definition schließt sich, dass eine richtige Parallelität nur dann existiert wenn mehrere Aufgaben auf der selben Anzahl an Prozessorkernen ausgeführt werden. (TODO: Hannes fragen ob dieser Schluss stimmt)

In dieser Thesis wird von Parallelität gesprochen wenn zwei Aufgaben auf zwei CPU parallel abgearbeitet werden. Von Concurrency wird in dieser Thesis gesprochen wenn mindestens zwei Tasks auf der CPU sequenziell oder abwechselnd abgearbeitet werden.


\subsection{Asynchrone Programmierung}
(TODO: write Section)

\subsection{Event getriebene Programmierung}
(TODO: write Section)
(TODO: write about dispatching and demultiplexing)
(TODO: write about callback functions)

\subsection{Lambdas}
Lambdas werden auch als Closures, anonyme Functions oder Blocks genannt. Lambdas sind Blöcke aus Quellcode welcher als Argument an eine Funktion übergeben werden kann. Unter anderem werden Lambdas in den folgenden Programmiersprachen unterstützt:

\begin{itemize}
  \item Lisp
  \item Ruby
  \item Javascript
  \item Java
  \item Smalltalk
\end{itemize}

Closures können auf lokale Variablen zugreifen. Das bedeutet dass eine Closure auf alle Variablen im aktuellen Kontext zugriff haben. In der Sprache Ruby werden Closures unter anderem mit geschwungenen Klammeren erstellt. \cite[]{fow04} 

\begin{lstlisting}[
	caption={\cite[]{fow04}},
	label=listing:closures
]
	File.open(filename) {|f| doSomethingWithFile(f)}
\end{lstlisting}

In Listing \ref{listing:closures} wird eine Datei vom lokalem Filesystem geöffnet. Die Funktion open in der File Klasse akzeptiert einen Block als Argument. Die Funktion open öffnet die Datei, führt den angegebenen Block aus und schließt nach Ausführung des Blocks die Datei. Blöcke können unter anderem in Transaktionen verwendet werden. \cite[]{fow04}


\subsection{I/O Operations}
(TODO: write Section)
(TODO: add statistiks about Disk I/O, Network I/O, Memory I/O)