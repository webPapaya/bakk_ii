\subsection{Futures}
\label{section:futures}

Ein Future \footnote{Im Javascript Bereich findet man Futures unter dem Begriff defferred} ist ein Platzhalter Objekt für die Berechnung eines Werts der in Zukunft verfügbar sein kann. Der Wert eines Futures wird asynchron im Hintergrund berechnet. Dabei darf der Wert eines Futures nur einmal gesetzt werden und im Anschluss nicht mehr verändert werden. Dadurch wird der Wert eines Future immutable. Möchte eine Funktion den Wert eines Futures erhalten so kann der Wert bereits verfügbar sein und wird somit sofort zurückgegeben. Wird der Wert noch berechnet so wird der Thread des aufrufers so lange blockiert bis der Wert verfügbar ist. Durch Futures können Werte parallel berechnet werden, was zu einem asynchronen, nicht blockierenden Programmfluss führt \cite[]{ScalaFutures}. 

\subsubsection{Funktionsweise} 

Ein Future ist ein Objekt welches die Berechnung eines Wertes representiert, der in Zukunft verfügbar sein kann. Dabei kann ein Future grundsätzlich drei Status besitzten: 

\begin{itemize}
  \item working
  \item completed
  \item error
\end{itemize}  
Besitzt das Future den Status \emph{working} so ist die Prozedur, welche den Wert des Futures berechnet noch nicht beendet. Sobald die Berechnung beendet ist wird der Wert des Futures auf das Resultat der Berechnung gesetzt. Der Status wird in diesem Schritt auf \emph{completed} gesetzt und der Wert wird in ein immutable verwandelt. Tritt bei der Berechnung ein Fehler auf wird der Status auf error gesetzt \cite[]{ScalaFutures}. 

Möchte eine Applikation auf den Wert eines Futures zugreifen so gibt es zwei Möglichkeiten wie sich die Applikation verhalten kann. Wenn der Status des Futures \emph{completed} ist, so wird der Wert der Variable direkt zurückgeliefert. Ist die Berechnung jedoch noch nicht beendet so wird der Thread des Aufrufers so lange blockiert, bis der Wert verfügbar ist \cite[]{ScalaFutures}.

In manchen Implementierungen kann man Callback Funktionen hinterlegen, welche aufgerufen werden sobald der Wert eine Futures verfügbar ist. 

\subsubsection{Schlüsse}
Futures bieten die Möglichkeit Prozeduren im Hintergrund berechnen zu lassen und diese bei bedarf mit dem Thread des Aufrufers zu synchronisieren. Wenn man ein Future als isolierten Container betrachtet, welcher keinen Status mit der Applikation teilt, dann sind Futures Thread-Safe und können keine Race Conditions beinhalten. Deadlocks sind bei Futures nicht auszuschließen, wenn zum Beispiel Future A auf den Wert von Future B wartet und umgekehrt.

\subsubsection{Implementierungen}
Futures können entweder als Sprachkonzept direkt in die Programmiersprache eingebettet sein oder als Bibliothek zu einer bestehenden Programmiersprache hinzugefügt werden. Die folgende Liste gibt einen Aufschluss über Programmiersprachen welche Futures direkt in die Programmiersprache integriert haben und erhebt keinen anspruch auf Vollständigkeit \footnote{Eine erweiterte Liste findet sich auf \url{http://en.wikipedia.org/wiki/Futures_and_promises#List_of_implementation}}:

\begin{itemize}
  \item Javascript
  \item Java
  \item Scala
  \item Dart
  \item Swift
\end{itemize}  

\subsubsection{Zusammenfassung}
Futures bieten die Möglichkeit Prozeduren wie Berechnungen oder I/O Operationen unabhängig vom Main Thread auszuführen. Dabei ist ein Future ein Platzhalter Objekt welches einen noch nicht brechneten Wert representiert. Ein Future kann drei unterschiedliche Status besitzen (working, completed, error). Futures können bei falscher Anwendung zu einem Deadlock führen. Wird mit einem Future kein globaler Status geteilt so können Race Conditions vermieden werden. 

Futures sind ein weit Verbreitetes Konzept und können zu Programmiersprachen welche keine Futures besitzen als Bibliotheken hinzugefügt werden.