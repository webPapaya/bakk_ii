\section{Zusammenfassung}
Ruby ist eine ausdrucksvolle Programmiersprache, welche grundsetzlich parallele Programmierung erlaubt. Die Referenzimplementierung CRuby kann durch GIL keine Operationen parallel ausführen. Dadurch werden zum einen häufig auftretende Probleme wie Race Conditions und Deadlocks verhindert, jedoch können diese dennoch auftreten. Yukihiro Matsumoto der Erfinder von Ruby hat bei der Rubycon 2012 in Denver auf die Frage nach der Zukunft von Concurrency in Ruby folgende Antwort gegeben:

	``
	I don't consider myself as the threading guy, so I don't think I can make the right decision about the Actor library or the threading library.
	'' (Yukihiro Matsumoto, Rubyconf 2012)

Daraus könnte man schließen dass in Zukunft die Bibliothek Celluloid oder eine andere Bibliothek in die Ruby Standard Library hineinfliesen könnte. Dieses Zitat ist jedoch aus dem Jahr 2012 und bis heute gibt es zum in der Ruby Community keine weiteren Bestrebungen das Thema Concurrency in Ruby grundlegend zu überdenken. Die Bestrebungen den GIL aus Ruby MRI mit einzelnen fein dosierten Locks zu ersetzen schlugen fehl, da dies zu einem Performance Verlust bei Single Threaded Applikation führte. 

Abschließend kann man sagen das man mehrere Operationenen in allen Ruby Interpreter gleichzeitig behandeln kann. Threads werden seit langer Zeit eingesetzt und werden von jedem relevanten Betriebssystem unterstützt. Auf dem ersten Blick sieht das verwenden von mehreren Threads relativ einfach aus. Probleme wie Race Conditions und Deadlocks können dabei relativ häufig auftreten und zu korrupten Daten oder einem Absturz der Applikation führen. Um diese Probleme zu lösen können Design Patterns wie das Reactor Patter, Actor Based Model oder Futures helfen. Für Concurrency in Ruby bietet sich das Projekt \emph{Concurrenc Ruby}\footnote{Genaue Informationen zu Concurrent Ruby\url{https://github.com/ruby-concurrency/concurrent-ruby}} an. In diesem Projekt findet sich eine Sammlung an unterschiedlichen Design Patterns für die Programmiersprache Ruby.

Ruby ist trotz aller bemühungen eine eher langsame Sprache \footnote{vgl. \url{http://benchmarksgame.alioth.debian.org/u32/compare.php}}. Je nach Interpreter können sich Zahlen bei einzelnen Algorithmen enorm unterscheiden. Aus diesem Grund gibt es Bestrebungen, Applikationen wie der CSS Preprocessor SASS, in einer schnelleren Sprache wie C neu zu implementieren. Für parallelisierbare Algorithmen welche keine oder nur wenige I/O Operationen durchführen empfiehlt es sich deshalb auf MRI zu verzichten und Rubinius oder JRuby zu verwenden. Ist diese Steigerung der Performance noch nicht genug empfiehlt es sich auf andere Sprachen wie Go, Scala oder Elixir auszuweichen welche ein besseres Concurrency Model besitzen. Bei der Entwicklung von Rails Applikationen spielt Concurrency für den Entwickler der Applikation nur eine untergeordnete Rolle, da Applikationsserver wie Unicorn oder Thin mehrere Prozesse erstellen über welche mehrere Anfragen parallel bearbeitet werden können.