\section{Ruby}
\label{section: Ruby}

In diesem Kapitel wird die Programmiersprache Ruby kurz vorgestellt. Dieses Kapitel ist nicht mit einer Einführung in Ruby vergleichbar, da eine solche über den Rahmen dieser Arbeit hinausgehen würde. Das Buch Programming Ruby \cite[]{Dav05} bietet eine Einführung in die Sprache kann als Referenz dienen. 

Auf der Website \footnote{\url{http://www.ruby-lang.org}} definiert sich Ruby wie folgt:

\begin{quote}
	Ruby ist eine dynamische, freie Programmiersprache, die sich einfach anwenden und produktiv einsetzen lässt. Sie hat eine elegante Syntax, die man leicht lesen und schreiben kann. \cite[]{Rub92}
\end{quote}



Im Kern ist Ruby eine rein Objekt Orientierte Programmiersprache. Dabei besteht ein Objekt aus einer Kombination von Status und Methoden welche diesen Status verändern. \cite[p. 1-2]{Dav05}

Ruby wurde im Jahr 2012 von der ISO (International Organization for Standardization) mit dem Standard ISO/IEC 30170 \footnote{Den Standard kann man der Website \url{http://www.iso.org/iso/home/store/catalogue_tc/catalogue_detail.htm?csnumber=59579} erwerben} standardisiert. 
