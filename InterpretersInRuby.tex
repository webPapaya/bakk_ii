\section{Ruby}
\label{section: Ruby}

In diesem Kapitel wird die Programmiersprache Ruby kurz vorgestellt. Dieses Kapitel ist nicht mit einer Einführung in Ruby vergleichbar, da eine solche über den Rahmen dieser Arbeit hinausgehen würde. Das Buch Programming Ruby \cite[]{Dav05} bietet eine Einführung in die Sprache kann als Referenz dienen. 

Auf der Website \footnote{\url{http://www.ruby-lang.org}} definiert sich Ruby wie folgt:

\begin{quote}
	Ruby ist eine dynamische, freie Programmiersprache, die sich einfach anwenden und produktiv einsetzen lässt. Sie hat eine elegante Syntax, die man leicht lesen und schreiben kann. \cite[]{Rub92}
\end{quote}

Im Kern ist Ruby eine rein Objekt Orientierte Programmiersprache. Dabei besteht ein Objekt aus einer Kombination von Status und Methoden welche diesen Status verändern. \cite[p. 1-2]{Dav05}

Ruby wurde im Jahr 2012 von der ISO (International Organization for Standardization) mit dem Standard ISO/IEC 30170 \footnote{Den Standard kann man der Website \url{http://www.iso.org/iso/home/store/catalogue_tc/catalogue_detail.htm?csnumber=59579} erwerben} standardisiert. Die Referenzimplementierung von Ruby ist CRuby oder MRI (Matz's Ruby Interpreter) \footnote{Der Quellcode ist auf Github verfügbar: \url{https://github.com/ruby/ruby}}.


\subsection{Concurrency in Ruby}

\subsubsection{Threads}
Ruby besitzt als Concurrency Model Threads. Um einen Thread in Ruby zu erzeugen erstellt man ein neues Thread Objekt welches einen Block akzeptiert.  

\begin{lstlisting}[language=Ruby,label=lst:thread_example]
def long_running
	sleep 2
	puts "second"
end
Thread.new { long_running }
puts "first"

# Output
#
# => first
# => second

\end{lstlisting}

In Listing \ref{lst:thread_example} sieht man ein Ruby Script welches einen Thread verwendet. Durch die Methode \emph{sleep} wird die Methode \textit{long\_run\_running} zwei Sekunden gestoppt. Im Beispiel sieht man dass die Zeile 6 als erstes auf dem Monitor ausgegeben wird als die Zeile 3. 

Die Klasse Thread bietet einige Methoden um mit Threads in Ruby zu arbeiten. \footnote{Eine genaue Dokumentation findet sich unter \url{http://ruby-doc.org/core-2.2.0/Thread.html}}

\subsubsection{Fibers}
Fibers sind leichtgewichtige Objekte in Ruby welche Concurrency ermöglichen. Dabei akzeptieren Fibers Code Blöcke welche von der Applikation gestartet, pausiert und weitergeführt werden können. Im Unterschied zu Threads wird der Zeitplan von Fibers nicht von der Ausführenden Umgebung (Virtuellen Maschine) bestimmt sondern vom Programmierer. 

Jede Fiber besitzt einen eigenen 4kb großen Call-Stack der das Pausieren eines Fibers ermöglicht. Fibers werden nicht automatisch gestartet und müssen manuell vom Programmierer gestartet werden.

\begin{lstlisting}[language=Ruby,label=lst:thread_example]
def long_running
	sleep 2
	puts "second"
end
Thread.new { long_running }
puts "first"

# Output
#
# => first
# => second

\end{lstlisting}



\subsection{Ruby Implementierungen}

Für die Programmiersprache Ruby gibt es unterschiedliche Implementierungen auf welche ich in diesem Kapitel eingehen möchte. Folgende Implementierungen werden in dieser Arbeit nicht behandelt da diese entweder nicht mehr aktiv Weiterentwickelt werden oder für diese Arbeit keine Relevanz besitzten:

\begin{itemize}
  \item MacRuby
  \item MRuby
  \item IronRuby
  \item MegLev
  \item Cardinal
  \item Ruby Enterprise Edition
\end{itemize}

Compiler wie Rubymotion \footnote{Cross Plattform Ruby Compiler für IOS und Android \url{http://www.rubymotion.com/}}, welche Ruby Code in eine andere Programmiersprache kompilieren, werden in dieser Thesis ebenfalls nicht besprochen.

\subsection{CRuby}


\subsection{JRuby}


\subsection{Rubinius}
Rubinius ist eine Implementierung von Ruby welche sich auf Concurrency spezialisiert hat. Dabei verwendet es native Threads um Ruby Code auf allen CPUs laufen zu lassen. Rubinius wird nicht interpretiert sondern besitzt einen JIT (Just In Time) Compiler. Dabei wird der Ruby Code in Bytecode transformiert, welcher von der Ribinius Virtual Maschine ausgeführt wird. Die virtuelle Maschine ist in C und C++ geschrieben und verwendet Teile wie den Ruby Parser von MRI. Die Sprache Ruby ist selbst in Ruby implementiert und kann auf der Virtuellen Maschine ausgeführt werden. Dadurch ist Rubinius eine Implementierung die zu einem großen Teil in Ruby implementiert ist. 

