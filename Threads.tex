\section{Prozesse und Threads}

Viele modernen Betriebssysteme können mehrere Operationen gleichzeitig ausführen. Das Ausführen eines Browsers und eines Textverarbeitungsprogramm stellt für moderne Betriebssysteme keine Schwierigkeit dar. Dieses Kapitel gibt einen Überblick über Prozesse und Threads. Um einen tiefen Einblick in dieses Kapitel zu erlangen bietet sich das Kapitel 2 aus dem Buch Moderne Betriebssysteme von Andrew S. Tanenbaum an. 

\subsection{Prozess}
\label{section: Prozess}
Tanenbaum beschreibt einen Prozess als ein Programm welches gerade ausgeführt wird. Ein Prozess beinhaltet den aktuellen Wertes des Befehlszählers, die Registerinhalte und die Belegung der Variablen. Um den Anschein einer parallelität für den BenutzerIn zu gewährleisten, wechselt die CPU zwischen einzelnen Prozessen.  Welches Programm wann ausgeführt werden darf entscheidet ein Scheduling Algorithmus. Dieser koordiniert die einzelnen Prozesse und gibt Rechenzeit auf der CPU für einen Prozess frei. Dadurch kann ein Betriebssystem, welches auf einem Computer mit Single Core Prozessor installiert ist, mehrere Dinge gleichzeitig erledigen. Zum Beispiel kann ein Textverarbeitungsprogramm ausgeführt werden und gleichzeitig ein Browser geöffnet sein \cite[p. 87]{tan09}.

\subsection{Threads}
\label{section: Threads}
Ein Prozess besitzt immer nur einen Ausführungsfaden. Mehrere Prozesse können parallel oder quasi-parallel abgearbeitet werden. Inter-Prozess Kommunikation erweist sich als problematisch da ein Prozess einen abgekapselten Adressraum besitzt. Es gibt Situationen in denen eine Applikation mehrere Ausführungsfäden parallel bearbeiten möchte. Threads bieten die Möglichkeit mehrere Ausführungsfäden in einem Prozess auszuführen. Das Ausführen von mehreren Threads in einem Prozess wird als Multithreading bezeichnet. Dabei besitzt ein Thread einen Befehlszeiger, ein Register und einen eigenen Stack. Der Befehlszeiger gibt an welcher Befehl als nächstes ausgeführt werden soll. Das Register beinhaltet alle lokalen Variablen. Im Stack werden aufgerufene Funktionen gestapelt welche noch nicht verlassen wurden. Durch den geteilten Adressraum zwischen den einzelnen Threads besitzen alle die selben globalen Variablen. Einzelne Threads können diese globalen Variablen lese, modifizieren und löschen \cite[p. 97]{tan09}. 

Bei der Realisierung von Threads kann man prinzipiell zwischen drei Arten unterscheiden\footnote{Auf Threads in Kernel und Hybride Realisierungen wird in dieser Arbeit nicht eingegangen. Eine Einführung in Kernel Threads und Hybride Threads findet sich in \cite[p. 109-110]{tan09}}:

\begin{itemize}
  \item Threads im Benutzeradressraum
  \item Threads im Kernel 
  \item Hybride Realisierung von Threads
\end{itemize}

Bei einer Implementierung von Threads im Benutzerraum weiß das Betriebssystem von den Threads nichts. Werden die Threads im Benutzerraum implementiert muss sich die Laufzeitumgebung um das Erstellen und Verwalten der Threads kümmern. Die Laufzeitumgebeung muss dafür eine eigene thread Tabelle verwalten in der alle Threads des Prozesses verwaltet werden. Die implementierung von Threads im Benutzeradressraum wird auch als Green-Threads bezeichnet \cite[p. 106]{tan09}.

(TODO: Refactor)
Blockierende Operationen wie das lesen eine I/O Operation kann von Threads im Adressraum realisiert werden, jedoch würde das alle weiteren Threads stoppen. Auf vielen UNIX Systemen gibt es die Funktion \emph{select}, welche überprüft ob eine Operation blockiert oder nicht. Wird die Operation blockiert, so wird ein anderer Thread ausgeführt. Wenn der blockierte Thread beim nächsten mal die Kontrolle bekommt, kann er ein weiteres mal überprüfen ob die operation blockiert. Dieser Vorgang wiederholt sich so lange, bis der Thread die Daten lesen kann \cite[p. 108]{tan09}. 


\subsection{Prozesse vs. Threads}

Prozesse und Threads teilen sich einige Eigenschaften. Beide Konzepte bieten die Möglichkeit Aufgaben parallel oder pseudo parallel zu bearbeiten. Prozesse teilen sich physische Ressourcen eines Computers wie Speicher, Festplatten, I/O Geräte, etc. Threads teilen sich den Adressraum und geöffnete Dateien eines Prozesses. Prozesse gehören immer zu genau einem Benutzer eines Betriebssystems. Threads hingegen gehören zu einem Prozess und besitzen dadurch nur indirekt einen Benutzer \cite[p. 89]{tan09}. 


\section{Probleme bei verwendung von Threads}
Bei der Verwendung von mehreren Threads in einer Applikation können durch den geteilten Adressraum auch einige Probleme auftreten. In vielen Fällen enstehen Probleme bei der synchronisation der Daten im Adressraum. Race Conditions und Deadlocks sind zwei häufig auftretende Probleme bei der Verwendung von mehreren Threads. 

\subsection{Context Switching}
Um einen möglichst faieren Zugriff auf die CPU für einzelne Threads zu bieten kann das Betriebssystem einzelne Prozesse zu jeder Zeit stoppen und zu einem späteren Zeitpunkt wieder ausführen. Der Zeitpunk wann ein Thread gestoppt werden soll und wann dieser wieder weiter ausgeführt werden soll wird von dem \emph{Scheduling Algorithmus} des Betriebssystems bestimmt. Die übergabe von CPU Ressourcen von einem Thread an den anderen wird als Context-Switch bezeichnet \cite[p. 23]{Sto2013}.

\subsection{Race Conditions}

Eine Race Condition tritt auf wenn mindestens zwei Threads (Thread A und B) auf die selbe Speicheradresse zugreifen und diese unerwartet verändern wird. Eine unerwartete Veränderung kann auftreten kann durch einen Context Switch entstehen. Entscheidet der Scheduling Algorithmus in einer kritischen Phase von Thread A, dass Thread B arbeiten darf kann Thread B auf die selben Werte von Thread A zugreifen. Thread B verändert einen Wert welcher von Thread A aktiv genutzt wird und nachdem Thread A seine arbeit wieder aufgenommen hat arbeitet dieser mit einem falschen Wert weiter \cite[p. 89]{tan09}. 

Ein Beispiel einer Race Condition zeigt Listing \ref{listing:race_conditions}. Dabei soll der Wert einer Variable ausgegeben werden, wenn dieser 10 ist.

\begin{lstlisting}[
	caption={},
	label=listing:race_conditions
]
if x == 10
  # Ein anderer Thread veraendert die Variable x auf 12
	puts x # Output ist 12 
end
\end{lstlisting}

In Listing \ref{listing:race_conditions} ist zu sehen, dass durch einen anderen Thread die Variable x auf einen anderen Wert gesetzt wird. Dadurch entsteht wird ein falscher Wert auf der Konsole ausgegeben. Dieses Verhalten ist in vielen Fällen nur schwer reproduzierbar, da der Scheduling Algorithmus schwer vorhersehbar ist \ref[]{Net92}.

\subsection{Deadlocks}

In einem Deadlock befinden sich mehrere Threads die jeweils auf eine Ressource warten, welche vom jeweilig anderen Prozess gesperrt ist \cite[p. 1]{Jah04}. Dadurch kann keiner der Threads einen fortschritt machen und bleiben in dem Zustand bis das Program terminiert. Deadlocks sind nicht nur auf Threads beschränkt sondern können auch vorkommen wenn mehrere Prozesse auf die Freigabe von Geräten, Datein oder Speicherstellen warten. Ein Deadlock kann auch PC übergreifend über ein Netzwerk entstehen \cite[p. 177]{tan09}. 

\subsubsection{The Dining Philosophers Problem}

Edsger Dijkstra beschreibt in seiner Thesis \emph{Hierarchical ordering of sequential processes} das Problem des Deadlocks anhand 5 Philosophen die gemeinsam an einem Tisch sitzen. Dabei beschreibt er folgendes Problem:

\begin{quote}
	Five pholosophers numberd 0 through 4 are living in a house where the table laid for them, each philosopher has his own place hat the table. Their only problem - besides those of philosophy - is that the dish served is very difficult kind of spaghetti, that has to be eaten with two forks. There are two forks next to each plate, so that presents no difficulty: as a consequence, however, no two neighbours may be eating simultaniously. \cite[p. 21]{dij71}
\end{quote} 

Alle Philosophen sind mit ihren Gedanken so beschäftigt, dass sie nicht miteinander kommunizieren können. Damit einer der Philosophen essen kann benötigt er beide Gabeln. Eine mögliche Reihenfolge der Operationen die nötig sind damit ein Philosoph essen kann lauten wie folgt \cite[p. 21]{dij71}:

\begin{itemize}
  \item Denke nach
  \item Ist der Philosoph nicht hungrig? Gehe zu Schritt 1
  \item Ist die rechte Gabel nicht frei, Gehe zu Schritt 1
  \item Nimm die rechte Gabel
  \item Ist die linke Gabel nicht frei, Denke Nach und wiederhole diesen Schritt
  \item Nimm die linke Gabel
  \item Esse
  \item Lege rechte Gabel auf den Tisch  
  \item Lege linke Gabel auf den Tisch
  \item Gehe zu Schritt 1
\end{itemize}

Das Problem bei dieser Implementierung ist die Möglichkeit eines Deadlocks. Dieses kann entstehen wenn alle Philosophen zur gleichen Zeit hungrig werden und die jeweils rechte Gabel in die Hand nehmen. Nachdem die jeweils linke Gabel vom rechten Philosophen bereits besetzt ist, denken alle Philosophen so lange nach bis die jeweilige linke Gabel frei wird. Nachdem alle Philosophen die rechte Gabel so lange für sich Beanspruchen bis sie nicht mehr hungrig sind. In diesem Zustand wird keiner der Philosophen jemals satt werden \cite[p. 21]{dij71}. 

Das \emph{Dining Philosophers Problem} kann auf die Informatik angewendet werden bei dem mehrere parallel laufende Prozesse auf geteilte Ressourcen warten welche vom jeweilig anderen Prozess blockiert wird.

\subsection{Zusammenfassung}
In diesem Kapitel wurden die Grundlagen von Prozesse und Threads näher erläutert. Prozesse können als Programme bezeichnet werden welche sich in Ausführung befinden. Dabei kann ein Betriebssystem mehrere Programme gleichzeitig ausführen. Dabei besitzen Prozesse einen linearen Programmfluss. Threads bieten die Möglichkeit diesen linearen Programmierfluss aufzubrechen und in einem gemeinsamen Adressraum parallele Prozeduren auszuführen. 

Durch den nichtlinearen Programmierfluss und das gleichzeitige behandeln mehrere Prozeduren im selben Adressraum können Race Conditions auftreten. Dabei werden Speicherstellen unerwartet verändert und führen zu korrupten Daten. Deadlocks sind ein weitere Probleme welche im Zusammenhang von paralleler Programmierung aufterten können. Diese treten auf, wenn mehrere Prozeduren auf eine Ressource warten, welche vom jeweilig anderen für sich alleine beansprucht wird. 