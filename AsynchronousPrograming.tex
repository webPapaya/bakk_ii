\section{Asynchrone Programmierung}

In der Literatur wird zwischen den Begriffen Blocking und Synchrone Programmierung und den Begriffen Non-Blocking und Asynchrone Programmierung oft keine differenzierung getroffen. In diesem Kapitel werden die Begriffe Blocking, Non-Blocking, Synchrone Programmierung und Asynchrone Programmierung näher erläutert. Als Grundlage für die Beschreibung der einzelnen Begriffe dient ein Beispiel aus der Informatik. 

Drei voneinander unabhängige Aufgaben sollen von einem Computer ausgeführt werden. Alle drei Aufgaben müssen abgeschlossen sein, damit das Program abgeschlossen ist \cite[]{Pet2015}. 

\subsection{Blocking vs Non-Blocking}

Beim behandeln von I/O Operationen kann eine Applikation dem Betriebssystem mitteilen wie die I/O Operation ausgeführt werden soll. Wird der Blocking Modus verwendet, so wartet das Program so lange bis alle Daten eingelesen wurden und bereit für die Weiterverarbeitung sind. Im Unterschied zum Blocking Mode wartet die Applikation nicht bis alle Daten eingelesen wurden. Die Funktion welche die I/O Operation ausführt leitet die Aufgabe (z.B. das einlesen einer Datei) an das Betriebssystem weiter und kehrt sofort zur aufrufenden Funktion zurück \cite[p. 47]{Erb2012}. Im Fall von NodeJS wird eine Callback Funktion hinterlegt, welche aufgerufen wird, wenn die I/O Operation abgeschlossen ist. Dadurch kann NodeJS in der Zwischenzeit andere Operationen ausführen. 

\subsection{Das synchrone Model}

Im synchronen Model werden Aufgaben sequentiell abgearbeitet. Dabei wird eine Initialaufgabe aufgerufen und von dieser ausgehend alle anderen Aufgaben sequentiell abgearbeitet. Ist eine Aufgabe fertig wird die nächste Aufgabe ausgeführt. Sind alle Aufgaben abgeschlossen so terminiert das Program. Der Programverlauf ist im synchronen Model Linear und kann in einer Zeitachse wie in Abbildung (TODO: add Image) dargestellt werden \cite[]{Pet2015}.

(TODO: add image)

\subsection{Das asynchrone Model}

Im asynchronen Model kehren Aufgaben sofort zum Ursprung zurück ohne dass die Aufgabe tatsächlich abgeschlossen ist. In der Zwischenzeit kann die Applikation weitere Aufgaben ausführen. 

\subsection{Das blockierenden und synchrone Model}
Im blockierenden und synchronen Model werden einzelne Operationen sequentiell abgearbeitet. Wird in diesem Model eine I/O Operation an das Betriebssystem geschickt, wird der Prozess so lange pausiert bis die I/O Operation abgeschlossen ist. Dannach wird der Programfluss weiter ausgeführt bis die nächste I/O Operation auftritt oder das Programm terminiert. Durch das blockierende und synchrone Model wird die CPU nicht optimal ausgelastet. In vielen Applikationen spielen I/O Operationen nur eine nebensächliche Rolle. Im Bereich der Webserver sind I/O Operationen jedoch ein wichtiger Bestandteil wodurch dieses Model für den Einsatz bei Webserver nur bedingt geeignet ist \cite[p. 48]{Erb2012}. 

\subsection{Das nicht blockierende synchrone Model}
Bei diesem Model werden I/O Operationen nicht blockiert. Das bedeutet, dass eine I/O Operation an das Betriebssystem gesendet wird und sofort zur aufrufenden Methode zurückkehrt. Nachdem die Daten der I/O Operation zum aktuellen Zeitpunkt noch nicht fertig eingelesen wurde muss die Applikation auf das einlesen warten. Nachdem die Applikation einen synchronen Programfluss besitzt, muss die Applikation auf das einlesen der Daten warten. Das nicht blockierende synchrone Model führt zu unnötigen Ressourcenverbauch da die Applikation zwar prinzipiell nicht blockiert, jedoch trotzdem auf die I/O Operation warten muss. Eine Möglichkeit der Implementierung wäre eine Schleife zu verwenden welche abbricht, wenn die I/O Operation abgeschlossen ist.\cite[p. 48]{Erb2012}

\subsection{Das blockierende asynchrone Model}
(TODO: write section, write about unix select(), poll(), etc.) 

\subsection{Das nicht blockierende asynchrone Model}
Beim nicht blockierenden asynchronen Model kehrt der I/O aufruf sofort an die aufrufende Funktion zurück. Ist die I/O Operation abgeschlossen wird entweder ein Ereignis getriggert oder eine Callback Funktion ausgeführt. Durch das nicht blockierende asynchrone Model wird die I/O Operation direkt im Kernel des Betriebssystems abgearbeitet. Dadurch kann die Applikation die Zeit in der die I/O Operation ausgeführt wird für andere Aufgaben nutzen \cite[p. 48]{Erb2012}.

Ein Grund warum das asynchrone nicht blockierende Model schneller sein kann als die beiden synchronen Modelle sind I/O Operationen die den Prozess blockieren\ref{subsection: io_operationen}. Angenommen eine Applikation möchte 2 Aufgaben erledigen bei der die erste Aufgabe eine Datei vom Dateisystem einliest. Im Falle der beiden synchronen Modelle müssten die Aufgabe 1 auf das fertige Einlesen der Datei warten. Dies kann entweder auf Kernelebene Funktionieren (= synchrones blockierendes Model) oder auf Applikationsebene (= synchrones nicht blockierendes Model) passieren \cite[]{Pet2015}. 

Im asynchronen nicht blockierenden Model kann die Applikation die Aufgabe 1 ausführen bis die operation blockiert. Während der Kernel die Datei einliest kann die Applikation die Aufgabe 2 ausführen und nach dem terminieren der Aufgabe 2 wird die Aufgabe 1 abgeschlossen. Dadurch macht das Program immer eine fortschritt ohne dass es auf auf Daten vom Kernel warten muss \cite[]{Pet2015}.

\subsection{Zusammenfassung}

In diesem Kapitel wurde das Prinzip der asynchronen Programmierung besprochen. Im Bezug auf asynchrone Programmierung kann man Programme in vier unterschiedliche Kategorien einteilen: 

\begin{itemize}
  \item blockierend und synchron
  \item nicht blockierend und synchron
  \item blockierend und asynchron
  \item nicht blockierend und asynchron
\end{itemize}    

Viele Anwendungen welche nur wenige I/O Operationen besitzen verwenden das blockierende und synchrone Model da es einen sequentiellen Programmierfluss erlaubt. Das nicht blockierende und synchrone Model verlagert das Warten auf die I/O Operation vom Betriebssystem in die Applikation, was zu einem unnötigen Ressourcenverbauch führen kann. 

(TODO: write about blockierend und asynchron)

Das asynchrone und nicht blockierende Model bietet die Möglichkeit die CPU bestmöglich auszunutzen da die Applikation während der I/O Operation eine andere Aufgabe ausführen kann.